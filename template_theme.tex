% Template for LaTeX slides
% Provide by E. de Saint Germain
% Last version: January, 2016

% Logo
\newcommand{\mylogo}{images/logo_argon_gray.png}
\newcommand{\coverlogo}{images/logo_argon_white.png}

% Second logo / Client Logo
\newcommand{\clientlogo}{images/logo_ponts.png}

% Image themme
\newcommand{\imagetheme}{images/Argon_theme.jpg}

% Margin
\newcommand{\xlmargin}{0.02\paperwidth}  % Left margin
\newcommand{\xrmargin}{0.98\paperwidth}  % Right margin

\newcommand{\headerheight}{0.95cm} % Header height
\newcommand{\footerheight}{0.7cm}  % Footer height

% Color theme
\newcommand{\maincolor}{argon orange} % Main color
\newcommand{\secondcolor}{argon gray} % Secundary color

% Format of classical slides
\newenvironment{myframe}[1][\insertsubsection]{\begin{frame}[plain]\startmyframe{#1}}{\stopmyframe\end{frame}}

\newcommand\startmyframe[1]{
  \begin{centering}
    \begin{tikzpicture}

      % Windows
      \clip (0,0) rectangle (\paperwidth,\paperheight);

      % Current coordinate
      \coordinate (current) at (0. ,8.25cm);

      % Title of the slide
      %\node[anchor = west, color = \maincolor] (slidename) at (\xlmargin, {\paperheight-.5*\headerheight}) {\textbf{#1}};
      \node[anchor = west, color = \maincolor] (slidename) at (\xlmargin, {\paperheight-.5*\headerheight}) {\begin{minipage}{.93\paperwidth}\textbf{#1}\end{minipage}};

      % Rectangle on the right
      \draw[color=\secondcolor, fill=\secondcolor] ({\paperwidth-.2cm},\paperheight) rectangle (\paperwidth,{\paperheight-\headerheight});
      \draw[color=\maincolor, fill=\maincolor] ({\paperwidth-.2cm},{\paperheight-\headerheight}) rectangle (\paperwidth,\footerheight);

      % Separator from the body of the slide
      \draw[color=\maincolor] (0,{\paperheight-\headerheight}) -- (\paperwidth,{\paperheight-\headerheight});

      % Line to help construction. Remove in final version.
      %\myhelplines
}

\newcommand\stopmyframe{
      % Separator from the body of the slide
      \draw[color=\secondcolor] (0,\footerheight) -- (\paperwidth,\footerheight);

      % Logos
      \node[anchor = west]  at (\xlmargin,{(\footerheight+.1cm)/2}) {\includegraphics[height={\footerheight-.25cm}]{\mylogo}};
      \node[anchor = east]  at ({\xrmargin},{(\footerheight+.1cm)/2}) {\includegraphics[height={\footerheight-.20cm}]{\clientlogo}};

      % Numbering pages
      \node[anchor = south] at (.5*\paperwidth,0) {\footnotesize \insertframenumber{} / \inserttotalframenumber};

      \draw[color=\maincolor, fill=\maincolor] (0,0) rectangle (\paperwidth,.1cm);

    \end{tikzpicture}
  \end{centering}

}


% Cover frame
% #1 title
% #2 subtitle
% #3 date
% #4 authors
\newcommand{\covermargin}{2cm}
\newcommand{\coverwidth}{.75\paperwidth}
\newcommand{\coverframe}[4]{
  \begin{frame}[plain]
    \begin{centering}
      \begin{tikzpicture}

        % Windows
        \clip (0,0) rectangle (\paperwidth,\paperheight);

        % Rectangles on the top
        \shade[left color=\secondcolor!90, right color=black!80] (0,\paperheight) rectangle (\paperwidth,{\paperheight-2.4cm});
        \draw[color=\maincolor, fill=\maincolor] (0,{\paperheight-2.4cm}) rectangle (\paperwidth,{\paperheight-2.9cm});

        % Logo
        \node[anchor = south east]  at ({\paperwidth-.9cm},{\paperheight-2.35cm}) {\includegraphics[height=0.5cm]{\coverlogo}};
        % Site web
        \node[anchor = east, color=white] (website) at ({\paperwidth-.9cm},{\paperheight-2.65cm}) {\footnotesize www.argon-consult.com};

        % Subtitle
        \node[anchor = west, color=\maincolor] (subtitle) at (\covermargin,{.33*\paperheight}) { \begin{minipage}{\coverwidth}\footnotesize \textbf{#2}\end{minipage} };
        % Title
        \node[anchor = south west, color=black] (title) at ($(subtitle.north west) + (0,.1cm)$) { \begin{minipage}{\coverwidth}\small\textbf{#1}\end{minipage} };
        % Date
        \node[anchor = north west, color=\secondcolor!60] (date) at ($(subtitle.south west) - (0,.1cm)$) { \begin{minipage}{\coverwidth}\footnotesize \textbf{#3}\end{minipage} };


        % Client logo / second logo
        \coordinate (website_SE) at ({\paperwidth-.9cm},{\paperheight-2.9cm});
        \coordinate (title_SE) at (title.north east);
        \coordinate (client_logo_E) at ($(website_SE)!.5!(title_SE)$);
        \node[anchor = east]  (client logo) at (client_logo_E) {\includegraphics[height=0.6cm]{\clientlogo}};

        % Authors
        \node[anchor = south, color=\secondcolor!60] (authors) at ({.5*\paperwidth},.5cm) {{\footnotesize \textbf{#4}}};

        % Rectangle at the bottom
        \draw[color=\maincolor, fill=\maincolor] (0,0) rectangle (\paperwidth,.1cm);

      \end{tikzpicture}
    \end{centering}
  \end{frame}
}



% Line to help construction. Remove in final version.
\newcommand\myhelplines{    
  \draw[help lines] (0,0) grid (\paperwidth, \paperheight); 
  \foreach \x in {1,...,12} { \node [anchor=south west, color=gray] at (\x,0) {\x}; }
  \foreach \y in {1,...,10} { \node [anchor=south west, color=gray] at (0,\y) {\y}; }
}



% Box to place latex frames
% #1 position, current by default
% #2 name of the box
% #3 width
% #4 tikz options
% #5 text
\newcommandx{\mytext}[5][1=current,2=box,usedefault]{
  \node[anchor = north west, #4] (#2) at (#1) {
    \begin{minipage}{#3}
      #5
    \end{minipage}
  };
  \coordinate (current) at (#2.south west);
}


% Table of content style
\defbeamertemplate{section in toc}{toc style}{%
  \leavevmode\leftskip=0.4cm%
  \llap{\textcolor{\maincolor}{$\bullet$}}%
  \kern1.5ex\textcolor{\secondcolor}{\inserttocsection}\par
}
\setbeamertemplate{section in toc}[toc style]

% Pop up the table of contents at the beginning of each section:
\AtBeginSection[]
{
  \begin{frame}[plain]
    \begin{centering}
      \begin{tikzpicture}

        % Windows
        \clip (0,0) rectangle (\paperwidth,\paperheight);

        % Headers
        \node[anchor = west, color = \maincolor] (outline) at (\xlmargin, {\paperheight-.5*\headerheight}) {\textbf{\outline}};

        % Separator from the table of content
        \draw[color=\maincolor] (0,{\paperheight-\headerheight}) -- (\paperwidth,{\paperheight-\headerheight});

        % Table of contents
        \mytext[\xlmargin,{\paperheight-2*\headerheight}]{.5\paperwidth}{anchor = north west, inner sep = 2pt
        }{
          \addtocounter{framenumber}{-1}
          \setcounter{tocdepth}{1}
          \renewcommand{\baselinestretch}{1.75}\normalsize
          \tableofcontents[currentsection]
          \renewcommand{\baselinestretch}{1.}\normalsize
        }

        % Separator from the body of the slide
        \draw[color=\secondcolor] (0,\footerheight) -- (\paperwidth,\footerheight);


        % Logo
        \node[anchor = west] at (\xlmargin,{(\footerheight+.1cm)/2}) {\includegraphics[height={\footerheight-.25cm}]{\mylogo}};

        % Rectangle at the bottom
        \draw[color=\maincolor, fill=\maincolor] (0,0) rectangle (\paperwidth,.1cm);

        % Vertical rectangle
        \draw[color=\maincolor, fill=\maincolor] (.55*\paperwidth,0) rectangle (.57*\paperwidth,\paperheight);
        % Image of the theme
        \node[anchor = west] at (.56*\paperwidth,.5*\paperheight) {\includegraphics[height=\paperheight]{\imagetheme}};

      \end{tikzpicture}
    \end{centering}
  \end{frame}
}

% Remove navigation symbols
\setbeamertemplate{navigation symbols}{}

% Define default itemization
\setbeamertemplate{itemize item}{\textcolor{\maincolor}{$\bullet$}}
\setbeamertemplate{itemize subitem}{\textcolor{\secondcolor!60}{{\scriptsize$\blacktriangleright$}}}
\setbeamertemplate{itemize subsubitem}{\color{black} \scriptsize $\bullet$}


%%% Local Variables:
%%% mode: latex
%%% TeX-master: "master"
%%% TeX-PDF-mode: t
%%% save-place: t
%%% End:
